\documentclass[12pt,a4paper]{article}
\usepackage[utf8]{inputenc}
\usepackage[english]{babel}
\usepackage[T1]{fontenc}
\usepackage{amsmath}
\usepackage{amsfonts}
\usepackage{amssymb}
\usepackage{graphicx}
\usepackage[left=2cm,right=2cm,top=2cm,bottom=2cm]{geometry}
\usepackage{setspace}
\usepackage{parskip}
\usepackage{natbib}
\usepackage{url}

\begin{document}

\setlength{\parindent}{20pt}

\begin{titlepage}
\author{Carlos Melian, Marcelo Awade, Charles Santana} % Correct names
\title{\textbf{Is long-term research vanishing?}}
\maketitle
\end{titlepage}

\doublespacing

\section*{General overview}
\begin{description}
 \item[1. General goals] \hfill
 
\textbf{Model the evolution of academia.}
How one or more academics traits evolve in time.
How the fitness of an academic researcher varies in time according to its publication or colaboration strategy (\textit{i.e.} point of view about science).
	
\item[2. Concepts] \hfill 
	
\textbf{Currency (\textit{i.e.} in terms of what fitness is evaluated)}: an index of publication strength; a publication score.
	
\textbf{Trait}: some variable describing an individual and used to determine their similarity with others.
% willingness of cooperation with another individual. (used to calculate the probability of cooperation).
%	\begin{description}
	%\item [\textit{Issues}] \hfill \\
	%What is exactly the trait? At this moment, I can only treat it as a continuous variable describing the personality of the individual. Or, a variable sinthesizing the subjects of interest of an individual. 
	%\end{description}
It may be interpreted as the personality of the individual, its behaviour, subjects of interest, point of view in science or any variable that sinthesizes an attribute of an individual. 
For example, a trait may describe an strategy of doing science, such as a gradient between broad interests and questions or deep interest in specific questions. 
Or, it may describe a gradient between empirical and theoretical interests.
Thus, trait is an abstract variable and here its value will vary between 0 and 1.

\textbf{Risk tolerance}: how the strength of interaction between two individuals varies in relation to the similarity between them. 
In this case, it is a parameter of the probability distribution giving the probability of cooperation as function of similarity.
For a given similarity between two individuals, those assuming higher risk have a higher probability of interaction compared to those assuming lower risk.
It should be interpreted as the willingness of making colaborations, in a way that more willingness is correlated to more tolerance risk, assuming that making a colaboration is always risky.


%\textbf{Gradient of strategies}: the strategies for colaboration and production can be seen as a gradient.
%At one extreme, there is a pure ``horizontal"  strategy (\textbf{H}) in which the individual looks for long-term research and for reconciling many subjects in an area of knowledge.
%At the other extreme, there is a pure ``vertical" strategy (\textbf{V}) in which the individual looks for fast and simple research that can be conducted in short time.

\item[\textbf{3. Assumptions}] \hfill

%\textbf{1)} As closer to the  \textbf{H} extreme as higher the risk tolerance. So individuals in the \textbf{H} extreme are more prone to colaborate with others no matter the similarity between them.
%Furthermore, \textbf{H} individuals produces articles in journals of higher impact factor than \textbf{V} strategists (any other measure of quality, impact or magnitude of the published research may be also used).
%However, \textbf{H} strategists spend more time to publish their works if compared to those in the \textbf{V} extreme of the gradient.

\item[4.]
$$ W_{ijk} ~ $$ (Environmental Selection, Tolerance to Risk, Administration Load)
$$ W_{ijk} = e^{-\alpha S_{ik}}(1+e^{-\beta S_{ij}} - \gamma)$$
$$ W_{ijk} = \phi N$$
$$ p_{ij} = e^{- \frac{1}{\beta_i}| t{i} - t{j}|}$$

$\beta_i$ is a parameter that measure the collaboration willingness of individuals according to their similarities.
\item[5.]

\item[6.]

\item[7.]

 \end{description}

\end{document}